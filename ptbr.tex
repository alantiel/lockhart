\documentclass[a4paper,oneside,12pt,notitlepage]{article}
\usepackage[utf8]{inputenc}
\usepackage[portuges]{babel}
\title{Lamentos de um Matemático}
\author{Paul Lockhart}

\frenchspacing
\linespread{1.3}

\begin{document}

\maketitle

Um músico acorda de um terrível pesadelo.
Em seu sonho ele se encontra numa sociedade onde a educação musical tornou-se obrigatória.
``Nós estamos ajudando nossos estudantes a se tornarem mais competitivos num mundo cada vez mais cheio de sons''.
Educadores, sistemas de ensino e o Estado são os responsáveis por esse importante projeto.
Estudos foram feitos, comitês foram formados e decisões foram feitas -- tudo sem a participação de um músico ou compositor.

Como músicos são conhecidos por registrarem suas ideias em forma de partituras, esses curiosos pontos e linhas pretas devem constituir a ``linguagem da música''.
É imperativo que os estudantes tornem-se fluentes nessa linguagem se eles almejam atingir qualquer grau de competência musical;
de fato, seria ridículo esperar que uma criança cantasse ou tocasse um instrumento sem ter uma boa base de notação e teoria musical.
Tocar e escutar música ou ficar sozinho compondo uma peça original são considerados tópicos muito avançados, por isso são geralmente adiados ao menos para a graduação e mais frequentemente para a pós-graduação.

No Ensino Fundamental o objetivo é treinar os estudantes no uso dessa linguagem -- manipular símbolos de acordo com um conjunto fixo de eregras:
``Aula de música é onde nós pegamos nosso caderno pautado, nosso professor coloca algumas notas no quadro e nós copiamos ou transpomos elas para um tom diferente.
Nós não podemos errar ao copiar as claves e as armaduras e nosso professor é muito exigente sobre pintar as semínimas completamente.
Uma vez tivemos um problema de escala cromática e eu acertei, mas o professor não me deu crédito porque eu coloquei as hastes no lado errado''.

No auge de sua sabedoria, os educadores logo perceberam que mesmo crianças muito novas podem receber este tipo de instrução musical.
De fato é considerado vergonhoso um garoto de terceira série não ter memorizado completamente o ciclo de quintas.
``Eu vou ter que levar meu filho a um professor particular.
Ele não se dedica aos seus deveres de música.
Diz que é chato.
Apenas senta lá, olha pela janela, cantarola para si mesmo e faz músicas bobas''.

Nas séries mais avançadas a pressão é mais forte.
Afinal, os estudantes precisam ser preparados para as tradicionais provas e para o vestibular.
Estudantes precisam ter aula de Escalas e Modos, Métrica, Harmonia e Contraponto.
``É bastante coisa pra aprender, mas depois na faculdade quando eles finalmente ouvirem tudo isso eles vão realmente apreciar o trabalho que fizeram no Ensino Médio''.
É claro que não são muitos estudantes que vão se dedicar à música, então apenas alguns irão escutar os sons que os pontos pretos representam.
De qualquer forma, é muito importante que todo membro da sociedade possa reconhecer uma modulação ou uma passagem fugal, não importa que eles nunca ouçam uma.
``Para dizer a verdade, a maioria dos estudantes não é muito boa em música.
Eles ficam entediados na sala, não tem habilidade e suas tarefas de casa são quase ilegíveis.
A maioria deles não se importa em como a música é importante no mundo de hoje;
eles apenas querem fazer o menor número possível de cursos de música pra terminar de uma vez.
Eu acho que existem pessoas que nasceram pra música e outras que não.
Eu tive uma criança que, cara, era sensacional!
Suas partituras eram impecáveis -- cada nota no lugar certo, caligrafia perfeita, sustenidos, bemóis, simplesmente lindos.
Ela será uma tremenda musicista um dia.''

Acordando e suando frio, o músico se dá conta que, felizmente, foi tudo apenas um sonho maluco.
``É claro!'' ele diz a si mesmo,
``Nenhuma sociedade reduziria uma forma de arte tão bonita e significativa para algo tão estúpido e trivial;
nenhuma cultura seria tão cruel com suas crianças a ponto de privá-las de uma expressão humana tão natural.
Que absurdo!''

Enquanto isso, do outro lado da cidade, um pintor acabara de acordar de um pesadelo semelhante\ldots

\end{document}

