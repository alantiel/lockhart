\documentclass[a4paper,oneside,12pt,notitlepage]{article}
\usepackage[utf8]{inputenc}
\usepackage[portuges]{babel}
\title{Lamentos de um Matemático}
\author{Paul Lockhart}

\frenchspacing
\linespread{1.3}

\begin{document}

\maketitle

Um músico acorda de um terrível pesadelo.
Em seu sonho ele se encontra numa sociedade onde a educação musical tornou-se obrigatória.
``Nós estamos ajudando nossos estudantes a se tornarem mais competitivos num mundo cada vez mais cheio de sons''.
Educadores, sistemas de ensino e o Estado são os responsáveis por esse importante projeto.
Estudos foram feitos, comitês foram formados e decisões foram feitas -- tudo sem a participação de um músico ou compositor.

Como músicos são conhecidos por registrarem suas ideias em forma de partituras, esses curiosos pontos e linhas pretas devem constituir a ``linguagem da música''.
É imperativo que os estudantes tornem-se fluentes nessa linguagem se eles almejam atingir qualquer grau de competência musical;
de fato, seria ridículo esperar que uma criança cantasse ou tocasse um instrumento sem ter uma boa base de notação e teoria musical.
Tocar e escutar música ou ficar sozinho compondo uma peça original são considerados tópicos muito avançados, por isso são geralmente adiados ao menos para a graduação e mais frequentemente para a pós-graduação.

No Ensino Fundamental o objetivo é treinar os estudantes no uso dessa linguagem -- manipular símbolos de acordo com um conjunto fixo de regras:
``Aula de música é onde nós pegamos nosso caderno pautado, nosso professor coloca algumas notas no quadro e nós copiamos ou transpomos elas para um tom diferente.
Nós não podemos errar ao copiar as claves e as armaduras e nosso professor é muito exigente sobre pintar as semínimas completamente.
Uma vez tivemos um problema de escala cromática e eu acertei, mas o professor não me deu crédito porque eu coloquei as hastes no lado errado''.

No auge de sua sabedoria, os educadores logo perceberam que mesmo crianças muito novas podem receber este tipo de instrução musical.
De fato é considerado vergonhoso um garoto de terceira série não ter memorizado completamente o ciclo de quintas.
``Eu vou ter que levar meu filho a um professor particular.
Ele não se dedica aos seus deveres de música.
Diz que é chato.
Apenas senta lá, olha pela janela, cantarola para si mesmo e faz músicas bobas''.

Nas séries mais avançadas a pressão é mais forte.
Afinal, os estudantes precisam ser preparados para as tradicionais provas e para o vestibular.
Estudantes precisam ter aula de Escalas e Modos, Métrica, Harmonia e Contraponto.
``É bastante coisa pra aprender, mas depois na faculdade quando eles finalmente ouvirem tudo isso eles vão realmente apreciar o trabalho que fizeram no Ensino Médio''.
É claro que não são muitos estudantes que vão se dedicar à música, então apenas alguns irão escutar os sons que os pontos pretos representam.
De qualquer forma, é muito importante que todo membro da sociedade possa reconhecer uma modulação ou uma passagem fugal, não importa que eles nunca ouçam uma.
``Para dizer a verdade, a maioria dos estudantes não é muito boa em música.
Eles ficam entediados na sala, não tem habilidade e suas tarefas de casa são quase ilegíveis.
A maioria deles não se importa em como a música é importante no mundo de hoje;
eles apenas querem fazer o menor número possível de cursos de música pra terminar de uma vez.
Eu acho que existem pessoas que nasceram pra música e outras que não.
Eu tive uma criança que, cara, era sensacional!
Suas partituras eram impecáveis -- cada nota no lugar certo, caligrafia perfeita, sustenidos, bemóis, simplesmente lindos.
Ela será uma tremenda musicista um dia.''

Ao acordar suando frio o músico se dá conta que, felizmente, foi tudo apenas um sonho maluco.
``É claro!'' ele diz a si mesmo,
``Nenhuma sociedade reduziria uma forma de arte tão bonita e significativa para algo tão estúpido e trivial;
nenhuma cultura seria tão cruel com suas crianças a ponto de privá-las de uma expressão humana tão natural.
Que absurdo!''

Enquanto isso, do outro lado da cidade, um pintor acabara de acordar de um pesadelo semelhante\ldots

\vspace{1em}

Eu me surpreendi ao me ver numa sala de aula convencional -- sem cavaletes, sem tubos de tinta.
``Ah, nós não pintamos antes do Ensino Médio'',
me disseram os estudantes.
``Passamos a sétima série estudando cores e aplicadores''.
Eles me mostraram uma planilha.
De um lado estavam amostras de cores separadas por espaços em branco.
Eles deveriam escrever seus nomes.
``Eu gosto de pintura'',
um deles fez questão de dizer,
``me dizem o que fazer e eu faço. É fácil!''

Após a aula eu fui falar com o professor.
``Então seus estudantes não fazem nenhuma pintura?''
perguntei.
``Bem, no próximo ano eles terão Pré-Pintura-por-Números.
% Há algum termo melhor pra Paint-by-Numbers?
Isso prepara eles para o Pintura-por-Números que terão no Ensino Médio.
Então eles poderão usar o que aprenderam aqui e aplicar em situações de pintura da vida-real -- mergulhando o pincel na tinta, limpando-o, coisas assim.
É claro que nós selecionamos nossos estudantes por habilidade.
Os que são realmente excelente pintores -- aqueles que sabem suas cores e pincéis até de trás pra frente -- podem pintar um pouco mais cedo e algum deles até assistem aulas de Localização Avançada por créditos de faculdade
% Essa frase ficou estranha e deve haver uma tradução melhor que Localização Avançada. Quem puder, modifique.
Porém, nós estamos apenas tentando dar a essas crianças uma boa base de o que é pintura, então quando eles saírem daqui para o mundo real e forem pintar sua cozinha eles não façam uma bagunça nela''.

``Hmmm, essas aulas no Ensino Médio que você mencionou\ldots{}''

``Pintura-por-números?
Nós estamos tendo muitas inscrições ultimamente.
Acho que a maioria vem por causa dos pais que querem garantir que suas crianças entrem em boas faculdades.
Nada parece melhor que Pintura-por-números Avançada num histórico de Ensino Médio''.
% Pode ser conveniente colocar uma nota de rodapé explicando como é a educação nos Estados Unidos.

``Por que as faculdades se importam com você conseguir preencher regiões numeradas com as cores correspondentes?''

``Ah, bem, você sabe, isso mostra um lúcido raciocínio lógico.
E é claro que se um estudante está planejando se formar numa ciência visual, como moda ou design de interiores, então é realmente uma boa ideia já sair do Ensino Médio com seus pré-requisitos relacionados a pintura''.

``Entendo.
E quando os estudantes podem pintar livres, numa folha em branco?''

``Você parece um de meus professores!
Eles sempre vem com essa de expressar os sentimentos e coisas desse tipo -- coisas abstratas que não tem realmente nada a ver.
Eu sou formado em Pintura e nunca trabalhei muito numa folha em branco.
Eu apenas uso os kits Paint-by-Numbers distribuídos pelo conselho escolar''.
% Convém revisar toda essa parte de pintura, não ficou muito boa.

\vspace{1em}

\end{document}

