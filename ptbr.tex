\documentclass[a4paper,oneside,12pt,notitlepage]{article}
\usepackage[utf8]{inputenc}
\usepackage[portuges]{babel}
\title{Lamentos de um Matemático}
\author{Paul Lockhart}

\frenchspacing
\linespread{1.3}

\begin{document}

\maketitle

Um músico acorda de um terrível pesadelo.
Em seu sonho ele se encontra numa sociedade onde a educação musical tornou-se obrigatória.
``Nós estamos ajudando nossos estudantes a se tornarem mais competitivos num mundo cada vez mais cheio de sons''.
Educadores, sistemas de ensino e o Estado são os responsáveis por esse importante projeto.
Estudos foram feitos, comitês foram formados e decisões foram feitas -- tudo sem a participação de um músico ou compositor.

Como músicos são conhecidos por registrarem suas ideias em forma de partituras, esses curiosos pontos e linhas pretas devem constituir a ``linguagem da música''.
É imperativo que os estudantes tornem-se fluentes nessa linguagem se eles almejam atingir qualquer grau de competência musical;
de fato, seria ridículo esperar que uma criança cantasse ou tocasse um instrumento sem ter uma boa base de notação e teoria musical.
Tocar e escutar música ou ficar sozinho compondo uma peça original são considerados tópicos muito avançados, por isso são geralmente adiados ao menos para a graduação e mais frequentemente para a pós-graduação.

\end{document}

